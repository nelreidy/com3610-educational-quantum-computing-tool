%%%%%%%%%%%%%%%%%%%%%%%%%%%%%%%%%%%%
\chapter{Introduction }
\label{chap:Introduction}
%%%%%%%%%%%%%%%%%%%%%%%%%%%%%%%%%%%%
\section{Quantum Computing Today}

Quantum computing is a rapidly emerging field at the intersection of physics, computer science, and mathematics, guaranteed to revolutionize industries such as cryptography, pharmaceuticals, and materials science. Quantum computing can solve certain problems exponentially faster than classical computers, completing tasks in seconds that would take classical machines years. However, quantum computing is still in its developmental stages, similar to where classical computers were decades ago. As Thomas G. Wong notes in \textit{Introduction to Classical and Quantum Computing}, while classical computers have become so abstracted and accessible that we can use and program them without an understanding of their inner workings, quantum computing has yet to reach this level of accessibility (Wong, 2020).
    


\section{Problems facing learners}
In my attempt to enter the field of quantum computing, I have observed some challenges an unguided self learner faces: 
\begin{itemize}
    \item \textbf{Lack of interactive educational material:} The only truly usable source I have found is IBM's Quantum Learning platform, specifically, their quantum composer. It is a fantastic platform in every way, but it is also quite overwhelming. I do not believe it is for beginners. 
    \item \textbf{Gaps in educational materials:} There are many sources available online. However, I have found that they are either too advanced, like the IBM platform, or they are too basic, never getting past the concepts of qubits, superposition and quantum gates -- at least before you hit a paywall. 
    \item \textbf{Lack of context and examples:} Many sources explain what quantum computing is, but  barely any explain how quantum computing came to be, or what it can be used for. The lack of concrete examples makes it difficult for a learner to fully appreciate the potential and practicality of quantum computing and algorithms.
    \item \textbf{Lack of guided learning material:} A few quantum simulators exist, but they either provide pre-built circuits or leave users to build their own without any guidance, making it hard for beginners to connect theory with practice.
    
\end{itemize}


\vspace{0.5cm}
\noindent {Based on my supervisor's project proposal and on my experience with current tools—where I have had to consult multiple books, online resources and simulators — I have established key specifications for a quantum computing educational system.}