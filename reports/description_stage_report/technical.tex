%%%%%%%%%%%%%%%%%%%%%%%%%%%%%%%%%%%%
\chapter{Project Overview}
\label{chap:Key_Contributions}
%%%%%%%%%%%%%%%%%%%%%%%%%%%%%%%%%%%%

\section{Project Aims}

This project aims to provide an interactive, guided platform that offers a comprehensive central learning experience for beginners in quantum computing. It will feature in-depth modules covering fundamental concepts, quantum circuits and algorithms, and their applications, alongside simulations and a virtual lab for hands-on practice.

\section{Project Specifications}
This educational system will be in the format of digestible course modules that contain quizzes,  assignments, and progress tracking to help learners evaluate their understanding and ensure continuous learning that can be paused and returned to. Modules will cover: 
\begin{itemize}
    \item Historical context of how quantum computing came to be, and the current state of the field.
    \item Fundamental quantum computing topics such as qubits, superposition, entanglement, quantum gates, quantum algorithms, and quantum error correction. 
    \item Practical exercises integrated within the modules, featuring guided circuit and algorithm execution using a built-in Qiskit development environment, with a strong focus on quantum algorithms. (A previous dissertation has covered gates and circuits.) Ideally, guidance would gradually decrease as modules progress. 
    \item Comparisons between classical and quantum algorithms, showcasing the efficiency of quantum computing. Algorithms such as Grover’s algorithm and Shor’s algorithm will be simulated.
    \item Possible advanced feature: saving and reusing of gates and algorithms. 
\end{itemize}

\vspace{0.3cm}
\noindent {The lab environment, through which exercises and guided learning are conducted, should also be available as an unguided tool for free experimentation.}

\section{Literature read to date}

To build a solid foundation in quantum computing, I am engaging with several resources. First, I am reading \textit{Introduction to Classical and Quantum Computing} by Wong (2022) and \textit{Quantum Computation and Quantum Information} by Nielsen and Chuang (2010), both of which provide insights into quantum computing fundamentals, algorithms, and error correction.  I am also slowly (but surely) exploring  \textit{Quantum Algorithm Implementations for Beginners} (J. et al., 2018), which delves into practical implementations of quantum algorithms using platforms like Qiskit. Additionally, I have taken advantage of online courses and tutorials available through IBM's Quantum Learning Platform (n.d.) and the Q-CTRL learning platform (n.d.), both of which offer interactive learning experiences for quantum computing concepts.