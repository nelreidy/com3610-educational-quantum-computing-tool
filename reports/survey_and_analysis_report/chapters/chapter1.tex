\chapter{Introduction}

Over the past two decades, numerous organizations and institutions worldwide have been striving to establish a foothold in quantum computing, aiming to "create the world's next big thing"\cite{Reference1}. This surge in interest has led to significant investments and a growing demand for accessible education in the field. \\
\newline
Understanding quantum computing requires not only a complex theoretical foundation but also the ability to grasp abstract concepts and apply them practically. This process often involves extensive simulation, as we are still in the early stages of quantum computing, with limited quantum computers available for commercial and educational use. Since quantum computing is a relatively new field, there is understandably a gap in accessible educational resources, making it challenging for learners to build a solid foundation without specialized guidance or a lengthy background in multiple disciplines. This is why an educational tool that bridges this gap is generally a good idea, and also in high demand. 

\section{Aims and Objectives}

This project aims to build onto an existing quantum circuit simulator to create an all-inclusive learning experience tailored for individuals with a computing background. It seeks to provide a strong foundational understanding of quantum computing, covering essential mathematics and quantum physics, making it possible for users to build knowledge and confidence in quantum computing without needing to constantly seek external resources. \\


This project specifically aims to focus on educational content for quantum algorithms, clearly explaining their uses and the underlying theory, while also simulating their execution and visualizing their circuits.\\


One major constraint of this project will be the limit to the complexity of circuits and algorithm simulations that the platform can handle as classical computers have inherent limitations when simulating quantum systems. This may not have a drastic effect as an entry level educational tool should be sufficient with material that can be classically simulated. Another possible limitation is the instability frequent breaking changes of quantum simulation packages and libraries like Qiskit. This poses the risk of the content becoming out of date. As such, it is important to consider the choice of library and what support/maintenance is needed to keep it up to date.





\section{Overview of the Report}
An overview of the rest of the report is as follows: 

\begin{itemize}
    \item \textbf{Chapter 2:} Literature survey to provide technical context, review of existing tools , as well as a technology survey with a discussion of the choice to continue  onto a previous student's project.  
    \item \textbf{Chapter 3:} Reflecting on chapter 2, this is a discussion of project specifications in detail, including design, implementation and testing. 
    \item \textbf{Chapter 4:} A review of what has been achieved up to date, with discussion of blockers and alterative implementations. 
    \item \textbf{Chapter 5:} A summary and a detailed plan of work until the end of project. 
\end{itemize}